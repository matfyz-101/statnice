\documentclass[a4paper]{report}
\usepackage{pdfpages}
\usepackage[utf8]{inputenc}
\usepackage[slovak]{babel}
\usepackage{graphicx}
\usepackage{xcolor}
\usepackage{environ}
\usepackage{ifmtarg}
\usepackage{listings}
\usepackage{lmodern}
\usepackage{rotating}
\usepackage{setspace}
\usepackage{etoolbox}
\usepackage{caption}
\usepackage{subcaption}
\usepackage[framemethod=default]{mdframed}
\usepackage{fullpage}
\usepackage{verbatim}
\usepackage{hyperref}

\definecolor{lightyellow}{rgb}{1.0,1.0,0.7}

\newcommand{\todoin}[1]{\fcolorbox{red}{red}{\color{yellow}\textbf{TODO: }}\fcolorbox{red}{lightyellow}{#1}}
\newcommand{\todo}[1]{\todoin{#1}\linebreak}
\newcommand{\bigtodo}[1]{\begin{mdframed}[frametitle={\color{yellow}\textbf{TODO:}},
 frametitlebackgroundcolor=red,
 backgroundcolor=lightyellow,
 skipbelow=2mm,
 linecolor=red,
 innertopmargin=.8\baselineskip] #1 \end{mdframed}}

\newcommand{\zadanieref}[1]{\refstepcounter{section}
\addcontentsline{toc}{section}{\protect\numberline{\thesection} #1}}

\newcommand{\zadanietpl}[2]{
\begin{mdframed}
\vspace{0.7em}
{\large \textbf \thesection} \quad {\large \textbf {#1}}
\nopagebreak
#2
\nopagebreak
\vspace{0.7em}
\end{mdframed}
\vspace{1 em}
}

\NewEnviron{zadanie}[1][]{%
\medbreak
\zadanieref{#1}
\nopagebreak
\zadanietpl{#1}{\vspace{1.25 em} \BODY}
}

\newcommand{\kzadanie}[1]{
\medbreak
\zadanieref{#1}
\nopagebreak
\zadanietpl{#1}{}
}

\begin{document}
\begingroup % predok
\pagenumbering{roman}
\tableofcontents
\endgroup % predok

\newpage
\pagenumbering{arabic}
\setcounter{page}{1}

\chapter{Databázy}

\begin{zadanie}[Účel databáz, charakteristika DB aplikácií, trojstupňová ANSI/SPARC architektúra, koncepčné dátové modely, navrhovanie databáz]

\begin{itemize}
 \item Entitno-relačný a relačný dátový model.
 \item Entitno-relačné diagramy, UML diagramy.
 \item Motivácia normalizácie.
\end{itemize}
\end{zadanie}

\begin{zadanie}[Teória navrhovania databáz]

\begin{itemize}
 \item Funkčné závislosti, Armstrongove axiómy.
 \item Uzáver množiny atribútov, uzáver množiny funkčných závislostí.
 \item Pokrytie a minimálne pokrytie množiny funkčných závislostí.
 \item Nadkľúče a kľúče.
 \item Relačné schémy, dekompozícia relačných schém, bezstratovosť dekompozície.
 \item Normálne formy: 3NF, BCNF.
 \item Naivná dekompozícia do 3NF a BCNF.
 \item Dekompozícia do 3NF zachovávajúca funkčné závislosti.
\end{itemize}
\end{zadanie}

\begin{zadanie}[Dotazovacie jazyky]
\begin{itemize}
 \item Relácie a predikáty, dotazy, relačný kalkul, Datalog, SQL, relačná algebra.
 \item Vyjadrovacia sila a vzájomné simulácie dotazovacích jazykov.
\end{itemize}
\end{zadanie}

\begin{zadanie}[Agregácia a rekurzia v dotazovacích jazykoch]
\begin{itemize}
 \item Grupovanie a agregácia v relačnej algebre, SQL, relačnom kalkule a Datalogu.
 \item Rekurzia v relačnej algebre, SQL, relačnom kalkule a Datalogu.
\end{itemize}
\end{zadanie}

\begin{zadanie}[Výpočet rekurzívnych programov a programov s negáciou]
\begin{itemize}
 \item Naivná evaluácia, seminaivná evaluácia.
 \item Unifikácia, SLD rezolúcia, rule-goal-tree, rule-goal-graph, väzby medzi premennými (binding patterns).
 \item Magická transformácia.
 \item Modely, stratifikovaná negácia, stable a well-founded sémantika.
\end{itemize}
\end{zadanie}

\begin{zadanie}[Optimalizácia dotazov]
\begin{itemize}
 \item Strom relačného výrazu.
 \item Optimalizácia konjunktívnych dotazov dekompozíciou hypergrafu, Wong-Youssefiho algoritmus.
 \item Optimalizácia pomocou semijoinov, úplný reduktor, Yannakakisov algoritmus.
\end{itemize}
\end{zadanie}

\begin{zadanie}[Optimalizácia konjunktívnych dotazov]
\begin{itemize}
 \item Pohltenie (containment) a ekvivalencia konjunktívnych dotazov.
 \item Testovanie pohltenia, petrifikované dotazy.
 \item Optimalizácia za predpokladu slabej ekvivalencie.
 \item Tablá a ich použitie.
\end{itemize}
\end{zadanie}

\begin{zadanie}[Transakcie]
\begin{itemize}
 \item Definícia transakcie, elementárne transakčné operácie, požiadavky na transakčný systém.
 \item Komponenty transakčného databázového systému.
 \item Rozvrhy, triedy rozvrhov.
 \item Konflikt-sériovateľnosť, testovanie konflikt-seriovatelnosti, view-sériovateľnosť, dvojfázové zamykanie, obnova.
 \item Striktné dvojfázové zamykanie, riešenie deadlockov.
 \item Časové pečiatky, validácia.
\end{itemize}
\end{zadanie}

\begin{zadanie}[Fyzická organizácia]
\begin{itemize}
 \item Fyzická algebra, zložitosť fyzických operátorov.
 \item Sekvenčné indexy, B stromy a B+ stromy.
 \item Hashovanie.
 \item Štruktúra hashovaného súboru: adresár, základné bloky, bloky preplnenia.
 \item Rozšíriteľné hashovanie.
 \item Lineárne hashovanie.
 \item Implementácia a zložitosť vybraných fyzických operátorov (merge-sort, nested-loop-join, ...).
\end{itemize}
\end{zadanie}

\begin{zadanie}[Distribuované databázy]
\begin{itemize}
 \item Atomický commit, výber koordinátora (bully algoritmus), replikácia dát.
 \item Distribuované zámky, distribuované deadlocky.
 \item Synchronizácia času, Christianov algoritmus, Berkeley algoritmus.
\end{itemize}
\end{zadanie}

\chapter{Distribuované systémy}

\begin{zadanie}[Vrstvové modely]
\begin{itemize}
 \item vrstvy, služby, rozhrania (interfaces), referenčný model OSI
 \item TCP/IP, úlohy jednotlivých vrstiev
\end{itemize}
\end{zadanie}

\begin{zadanie}[Sieťová vrstva v TCP/IP]
\begin{itemize}
 \item adresácia v TCP/IP, protokoly IP, ARP, ICMP
 \item routovanie (smerovanie) v TCP/IP, NAT
\end{itemize}
\end{zadanie}

\begin{zadanie}[Transportná vrstva v TCP/IP]
\begin{itemize}
 \item úlohy a služby transportnej vrstvy
 \item protokoly UDP a TCP
\end{itemize}
\end{zadanie}

\begin{zadanie}[Fyzická a linková vrstva]
\begin{itemize}
 \item Ethernet, CSMA/CD
 \item rozširovanie Ethernetu na fyzickej a linkovej vrstve – huby, switche, WiFi, CSMA/CA
\end{itemize}
\end{zadanie}

\begin{zadanie}[Domain Name System (DNS)]
\begin{itemize}
 \item doménové meno, úloha DNS, architektúra DNS
 \item typy záznamov, reverzné vyhľadávanie
\end{itemize}
\end{zadanie}

\begin{zadanie}[Bezpečnostné mechanizmy na linkovej vrstve]
\begin{itemize}
 \item VLAN
 \item riadenie prístupu k portu na báze linkovej adresy
 \item IEEE 802.1X, bezpečnosť WiFi
\end{itemize}
\end{zadanie}

\begin{zadanie}[Bezpečnostné mechanizmy na sieťovej a transportnej vrstve]
\begin{itemize}
 \item firewall
 \item IPSec
 \item SSL/TLS
\end{itemize}
\end{zadanie}

\begin{zadanie}[Bezpečnosť elektronickej pošty a webu]
\begin{itemize}
 \item problémy a riešenia, end-to-end security
 \item PGP, S/MIME
 \item komunikácia so serverom (IMAPS, POP3S, SMTPS)
 \item HTTPS, certifikáty
\end{itemize}
\end{zadanie}

\begin{zadanie}[Model komunikácie so zdieľanou pamäťou]
\begin{itemize}
 \item POSIX thready, vytváranie, ukončovanie a finálna synchronizácia threadov.
 \item Zdieľané premenné, mutex, conditional variable.
 \item Scheduling.
\end{itemize}
\end{zadanie}

\begin{zadanie}[Thready v Jave]
\begin{itemize}
 \item Vytváranie threadov, ukončovanie threadov.
 \item Synchronizácia prístupu k metódam a premenným.
 \item Synchronizácia na úrovni tried a na úrovni objektov.
 \item Scheduling.
\end{itemize}
\end{zadanie}

\begin{zadanie}[Kanálový model komunikácie]
\begin{itemize}
 \item Synchrónna a asynchrónna kanálová komunikácia.
 \item sémantika posielania a prijímania správ.
\end{itemize}
\end{zadanie}

\begin{zadanie}[Point-to-point model komunikácie]
\begin{itemize}
 \item Synchrónna a asynchrónna point-to-point komunikácia.
 \item Sémantika posielania a prijímania správ.
\end{itemize}
\end{zadanie}

\begin{zadanie}[Problém Commitu v distribuovaných databázach]
\begin{itemize}
 \item Atomický commit, výber koordinátora (bully algoritmus).
\end{itemize}
\end{zadanie}

\begin{zadanie}[Replikácia dát v distribuovaných databázach]
\begin{itemize}
 \item Replikácia dát.
 \item Distribuované zámky, distribuované deadlocky.
\end{itemize}
\end{zadanie}

\begin{zadanie}[Synchronizácia času v distribuovaných systémoch]
\begin{itemize}
 \item Synchronizácia fyzických hodín, Christianov algoritmus, Berkeley algoritmus.
 \item Lamportov čas.
\end{itemize}
\end{zadanie}

\chapter{OO analýza}

\kzadanie{Use-case modelovanie}
\kzadanie{Modelovanie tried}
\kzadanie{Modelovanie kompozitných štruktúr}
\kzadanie{Modelovanie interakcií}
\kzadanie{Modelovanie stavových automatov}
\kzadanie{Modelovanie aktivít}
\kzadanie{Modelovanie komponentov}
\kzadanie{Modelovanie nasadenia systému}
\kzadanie{Pomocné modelovacie konštrukty}
\kzadanie{UML profily}

\chapter{Teória paralelných výpočtov}

\begin{zadanie}[Paralelné gramatiky]
\begin{itemize}
 \item paralelné prepisovanie
 \begin{itemize}
  \item Lindenmayerove systémy
  \item Indické a Ruské paralelné gramatiky
  \item generatívne systémy
 \end{itemize}
 \item kooperujúce distribuované gramatiky
 \item paralelné gramatické systémy
\end{itemize}
\end{zadanie}

\begin{zadanie}[Paralelné stroje]
\begin{itemize}
 \item alternujúce Turingove stroje
 \item alternujúce konečné automaty
 \item boolovské obvody
 \item PRAM
 \item počítače druhej triedy
\end{itemize}
\end{zadanie}

\begin{zadanie}[Ťažké problémy]
\begin{itemize}
 \item Trieda NC
 \item redukovateľnosť
 \item P-úplné problémy
\end{itemize}
\end{zadanie}

\chapter{Výpočtová zložitosť}

\begin{zadanie}[Turingove stroje, základné zložitostné triedy a vzťahy medzi nimi]
\begin{itemize}
 \item Savitchova veta
 \item simulácie a hierarchie
\end{itemize}
\end{zadanie}

\kzadanie{Gap Theorem a Veta o zrýchľovaní}
\begin{zadanie}[NP-úplnosť, Cook-Levinova veta]
a niektoré ďalšie (aj pre prax dôležité) NP-úplné problémy
\end{zadanie}
\kzadanie{Vzťah NP-úplných a NP-optimalizačných problémov}
\kzadanie{Aproximačné algoritmy pre NP-optimalizačné problémy; neaproximovateľnosť}

\begin{zadanie}[Pravdepodobnostné algoritmy]
\begin{itemize}
 \item Veta o vylepšovaní
 \item BPP vs. PSPACE
\end{itemize}
\end{zadanie}

\chapter{Vypočítateľnosť}

\begin{zadanie}[Churchova-Turingova téza]
\begin{itemize}
 \item Zdôvodnenie ekvivalencie Turingovych strojov počítajúcich funkciu a Minského registrových strojov.
\end{itemize}
\end{zadanie}

\begin{zadanie}[Primitívna rekurzia]
\begin{itemize}
 \item Definícia tejto triedy funkcií a základné vlastnosti.
 \item Súvis s programami bez while-cyklov.
\end{itemize}
\end{zadanie}

\begin{zadanie}[Rekurzívne a čiastočne rekurzívne funkcie]
\begin{itemize}
 \item Definícia a ekvivalencia s automatovými modelmi vypočítateľnosti.
 \item Rekurzívne množiny a predikáty.
\end{itemize}
\end{zadanie}

\begin{zadanie}[Aritmetizácia syntaxe]
\begin{itemize}
 \item Kódovanie objektov do čísel.
 item Univerzálne funkcie a ich zložitosť vzhľadom na príslušnú triedu funkcií.
\end{itemize}
\end{zadanie}

\begin{zadanie}[Vyčísliteľné reálne čísla]
\begin{itemize}
 \item Definícia a základné vlastnosti.
\end{itemize}
\end{zadanie}

\begin{zadanie}[Nerozhodnuteľné problémy]
\begin{itemize}
 \item Funkcia Busy Beaver.
 \item Problém zastavenia a jeho úplnosť pri many-to-one redukcii.
\end{itemize}
\end{zadanie}

\chapter{Vyhľadávanie v texte}

\begin{zadanie}[Vyhľadávanie vzorky v texte]
\begin{itemize}
 \item pomocou konečných automatov
 \item Knuth-Morris-Prattov algoritmus
\end{itemize}
\end{zadanie}

\kzadanie{Sufixové stromy a príklady ich použitia}
\kzadanie{Sufixové polia}
\kzadanie{Výpočet editačnej vzdialenosti medzi reťazcami}
\kzadanie{Výpočet najdlhšej spoločnej podpostupnosti dvoch reťazcov}
\kzadanie{Vyhľadávanie približných výskytov vzorky v texte}

\chapter{Kompilátory}

\begin{zadanie}[Štruktúra kompilátorov]
\begin{itemize}
 \item Základné pojmy. Vzťah k teórii jazykov. Syntaktická analýza. Sémantika. Jedno a viac prechodové kompilátory.
 \item Jednoduchý jednoprechodový kompilátor. Rekurzívny zostup. Syntaktické diagramy.
\end{itemize}
\end{zadanie}

\begin{zadanie}[Lexikálna analýza]
\begin{itemize}
 \item Dôvody oddelenia lexikálnej a syntaktickej analýzy. Lexikálna štruktúra jazykov. Regulárne jazyky. Konečné automaty.
 \item Syntéza: Regulárny jazyk - nedeterministický konečný automat - deterministický konečný automat.
\end{itemize}
\end{zadanie}

\begin{zadanie}[Syntaktická analýza]
\begin{itemize}
 \item Bezkontextové gramatiky. Návrh gramatiky jazyka. Syntaktické stromy.
 \item Metódy zhora - dolu: rekurzívny zostup, LL(1), LL(k).
 \item Metódy zdola - hore: Operátorovo precedenčné gramatiky a jazyky, LR(k) metódy ( SLR(1), LALR(1)).
 \item Oprava chýb a zotavenie sa z chýb pri syntaktickej analýze.
\end{itemize}
\end{zadanie}

\begin{zadanie}[Syntaxou riadené preklady]
\begin{itemize}
 \item Atribútové gramatiky. Syntetizované a zdedené atribúty.
 \item Vyhodnocovanie atribútov. Porovnanie metód zhora - dole a zdola - hore.
 \item Graf závislosti atribútov.
\end{itemize}
\end{zadanie}

\begin{zadanie}[Kontrola typov]
\begin{itemize}
 \item Systém typov v programovacích jazykoch a konverzie medzi typmi.
 \item Gramatika typov a kontrola typov ako syntaxou riadený preklad.
 \item Polymorfické typy. Unifikácia.
\end{itemize}
\end{zadanie}

\begin{zadanie}[Podpora v čase behu (Run-time environments)]
\begin{itemize}
 \item Organizácia a prideľovanie pamäte.
 \item Odovzdávanie parametrov, lokálne a nelokálne premenné.
 \item Prideľovanie a organizácia dynamickej pamäte.
 \item Tabuľka symbolov - hašovanie.
\end{itemize}
\end{zadanie}

\begin{zadanie}[Generovanie medzijazyka]
\begin{itemize}
 \item Formy medzijazyka:
 \begin{itemize}
  \item Poľská bezzátvorková forma
  \item trojadresový kód
  \item trojice, štvorice
 \end{itemize}
 \item Generovanie medzijazyka syntaxou riadeným prekladom.
 \item Popis základných konštrukcii: deklarácie, výrazy, príkaz priradenia, booleovské výrazy, podmienené príkazy a cykly. Volanie procedúr a funkcií.
 \item Spätné plátanie (backpatching).
\end{itemize}
\end{zadanie}

\begin{zadanie}[Generovanie kódu]
\begin{itemize}
 \item Cieľový počítač: CISC, RISC, SIC. Ceny inštrukcií.
 \item Administrácia pamäte. Statická pamäť, zásobník, dynamická pamäť - heap.
 \item Prideľovanie registrov. Prideľovanie registrov farbením grafu.
 \item Generovanie kódu z dagu.
 \item Generovanie kódu pokrývaním ozdobeného stromu.
\end{itemize}
\end{zadanie}

\begin{zadanie}[Optimalizácia kódu]
\begin{itemize}
 \item Kukátková (peephole) optimalizácia.
 \item Optimalizácia základných blokov, optimalizačné transformácie.
\end{itemize}
\end{zadanie}

\chapter{Kolmogorovská zložitosť}

\begin{zadanie}[Definícia, význam a základné vlastnosti Kolmogorovskej zložitosti]
\begin{itemize}
 \item Varianty Kolmogorovskej zložitosti a vzťahy medzi nimi
\end{itemize}
\end{zadanie}

\begin{zadanie}[Nestlačiteľné reťazce]
\begin{itemize}
 \item definícia, význam.
 \item Vysvetlenie metódy nestlačiteľnosti a ukážka jej využitia na konkrétnych príkladoch.
\end{itemize}
\end{zadanie}

\end{document}